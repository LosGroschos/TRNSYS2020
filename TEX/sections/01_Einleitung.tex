\section{Einleitung}

In dem Praktikumsversuch Meteorologische Messgrößen sollen verschiedene Wetterdaten aufgenommen und die Funktionen und Messprinzipien der jeweiligen Geräte erklärt werden.
Die aufgezeichneten Wetterdaten beinhalten die Globalstrahlung mit Anteilen der Direkt- und Diffusstrahlung, die Umgebungstemperatur und Luftfeuchte, die Windrichtung und -geschwindigkeit sowie eine Niederschlagsmessung. Der Luftfeuchtesensor soll anhand zweier Referenz-Salzlösungen kalibriert und mit der Umgebungsluftfeuchte verglichen werden. Die verschiedenen Strahlungsmessgeräte sollen auf ihr Ansprechverhalten und die Genauigkeit überprüft werden. Die Niederschlagsmessung wird mit einer Wasserflasche simuliert und dient lediglich dem Verständnis des Messprinzips.

